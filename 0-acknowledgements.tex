\chapter*{Remerciements}
% \vspace{-20pt}

Cette thèse aura été une belle aventure pour moi, elle m’a permis de m’enrichir d’un point de vue personnel et scientifique, en m’apportant de nouveaux regards sur le monde qui m’entoure. Même si son parcours a été parsemé d’obstacles, j’ai appris qu’il est toujours possible d’aller plus loin et de trouver un (ou même plusieurs) moyen(s) pour contourner les difficultés. Une thèse ne se fait pas tout seul, c’est pourquoi je tiens à remercier beaucoup de personnes qui m’ont soutenu jusqu’à son terme (j’espère ne pas en oublier).

Une thèse ne serait rien sans encadrants et sur ce point, j’ai eu beaucoup de chance d’avoir été encadré par Marcelo Dias de Amorim et Prométhée Spathis. C’est vrai que c’est un travail sur le long terme qui peut par de nombreux aspects s’apparenter à un rallye, avec ses routes tortueuses, ses dénivelés et épingles à cheveux. Mais avec un bon 4x4 (enfin 3x3 ici), on arrive à tout faire. Merci pour m’avoir guidé tout au long de cette belle expérience qui m’a permis de mûrir scientifiquement, mais aussi humainement.

Je voulais aussi remercier Sidi-Mohammed Senouci et Marco Fiore qui ont bien voulu commenter ma thèse. Je suis conscient qu’ils ont passé beaucoup de temps à la lire et à rédiger leur rapport. Leurs remarques m’ont permis d’améliorer la qualité du manuscrit et de la présentation. Je tenais aussi à remercier Hubert Tardieu, André-Luc Beylot, Olivier Festor et Serge Fdida pour avoir accepté de faire partie de mon jury et évaluer les travaux que je propose.

Je voulais aussi remercier toutes les personnes avec qui j’ai eu la chance de collaborer. J’ai entamé mes premières collaborations avec Hervé Rivano dès mon stage de Master, avec qui j’ai pu apprendre beaucoup sur les techniques d’optimisation. Je voulais aussi remercier Luis Costa et Miguel Campista pour m’avoir permis de participer à leurs recherches sur la virtualisation. I had the pleasure of meeting Yannis Viniotis during his summer visit in Paris. Thank you for all of the discussions we had and the time that you spent helping me improving my work. I was also very fortunate to have the opportunity to visit Mostafa Ammar in the US. It was truly a memorable experience in which I learned a great deal about the American style of research in the beautiful state of Georgia. Thank you for this opportunity and all the insightful discussions we had. I also wanted to thank all my colleagues in GeorgiaTech for the many discussions we’ve had and for welcoming me in your lab: Tarun, Ahmed, Karim and Samantha.

Je remercie tous mes collègues du LIP6, les permanents (avec une pensée pour Marguerite) et les doctorants, en commençant par ceux qui étaient déjà au LIP6 quand j’ai commencé et qui sont partis (ou presque)~: Tiphaine, Alexandru, Filippo, Marco, Jordan, Ahlem, Fadwa, Raul, Thiago et Matteo (félicitations !)~; ceux qui ont commencé avec moi et qui finissent maintenant (ou presque)~: Alexandre (depuis la L1 !), Quentin, Giulio, Davide, Antonella, Rudyar et Loïc~; et enfin ceux qui vont rester (bon courage !)~: Minh, Mustafa, Narcisse, Florian, Amr et Salah.

Je tenais aussi à remercier mes amis qui, depuis longtemps, ont subi mes monologues un peu trop \textit{nerd} ou \textit{geek} à leur goût. Donc, merci à Christophe, Sébastien, Baptiste, Cyril, Alexis, Laurent, Hubert et Philippe.

Enfin, je voulais remercier ma maman, mon papa, ma soeur, Djeamson, et surtout Nohemi qui ne m’a pas seulement supporté tout au long de cette thèse, mais qui m’a aussi beaucoup \textit{supported} (hehe) dans cette aventure. And for this, I am super grateful!


