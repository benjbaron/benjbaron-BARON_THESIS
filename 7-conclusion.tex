\chapter{Conclusion \& perspectives}

% INTRODUCTION - SUMMARY OF FINDINGS

We proposed a solution that leverage the vehicles in circulation to offload large amounts of data from conventional data networks such as the Internet. Offloading represents an effective solution to relieve the Internet from part of this traffic. As an example, large cloud providers such as Amazon Web Services offer their customers to ship through postal services large amounts of data to cloud services.

Our offloading service exploits the movements of the vehicles to transport data between facilities equipped with storage or offloading spots located on the way of vehicles where they stop long enough to transfer large amounts of data, including parking lots, shopping centers, and gas stations. The offloading spots buffer the data carried by the vehicles when they make stops along their way. The offloading spots act as relay to combine the different trajectories of vehicles together and asynchronously pass the data from one vehicle to another. 

The focus of this thesis was to efficiently allocate demands to offload reliable data transfers using the movements of the vehicles. The complexity of the road network and the number of trips make the allocation intractable, so we created the offloading overlay to characterize the trips traveled between the offloading spots by networking quantities in terms of capacity, delay, and loss. We leverage this simpler representation of the road network to formulate data transfer allocation as different but complementary linear programming models. In the feasibility assessment of our solution, our model maximizes the cost-benefit of offloading traffic compared to using conventional data networks. In the implementation, we relied on a centralized architecture with a holistic view of the system to manage and configure the offloading spots to match the requirements of the offloading demands. The controller uses a max-min fairness model to guarantee fair allocation of competing demands and efficient utilization of the vehicular resources. In both models, we implement mechanisms at the offloading spots to ensure reliable data transfers. In the feasibility case, we replicate the data between two offloading spots enough to recover a maximum proportion of missing data. In the implementation case, we retransmit the data that failed to be delivered to the expected offloading spots and use redundant copies to decrease the amount of data to retransmit. In both studies, we show that the road network can accommodate concurrent data transfers with an aggregate capacity in the Petabyte range per week.

We extended the concept of offloading spots in two directions to implement vehicular cloud services in urban scenarios. In both extensions, we studied the optimal placement of the offloading spots to efficiently combine the movements of the vehicles that match the service requirements. Firstly, we exploited the storage of the offloading spots to turn them into repositories that are part of a distributed cloud-like storage and sharing system. The system relies on the movements of the vehicles between the repositories to distribute and replicate the files users upload in the system. The placement algorithm determines the optimal locations of the repositories to jointly satisfy the users requests to store or retrieve a file and distribute the data with the movements of the users between the repositories. Secondly, we dematerialized the offloading spots and represented them by areas where vehicles often come in contact long enough to transfer large amounts of data. To select the areas, we used a logical representation of the locations of the contacts and their densities according to different durations. In both cases, we showed the benefits of the placement strategies to either distribute files and satisfy user requests in a timely fashion, or transfer large amounts of data using bus transportation systems.

% The development of data-intensive services cause ever-growing amounts of data exchanges on the Internet, including background traffic between geo-distributed data centers. Traffic offloading represents an effective solution to relieve the Internet from part of this traffic. It creates an alternative ways to transport the data, other than using legacy infrastructure networks. Offloading is notably offered by large cloud providers such as Amazon Web Services to allow their customers to send large amounts of data to cloud services.

% In this thesis, we have studied a system that offloads traffic using the vehicles in circulation to transport the data. This system exploits the potential bandwidth resulting from the 40 billions of trips traveled every year in France alone. Instead of relying on the vehicles making the trip from the source to the destination of the data transfers, our offloading system relies on dedicated facilities that act as relays where vehicles take turns to transport data towards its destination. The facilities are equipped with data storage, allowing the vehicles to load and unload data when they stop along their way at these facilities. 


% \section{Summary of contributions}

% \paragraph{Centralized architecture.}

% Throughout this thesis, we relied on a centralized controller that has a holistic view of the system to manage and configure the offloading spots to match the requirements of demands to offload data. This architecture enables the efficient allocation and reliability of data transfers over the road network. We showed the benefits of the centralized architecture in terms of resulting throughput over local forwarding decisions taken by the offloading spots.

% \paragraph{Offloading overlay.}

% The complexity of the topology of the road network and the large number of trip travelled make the allocation problem computationally intractable. We introduced the offloading overlay as an abstract representation of the resources of the road network to mitigate its scale. The nodes of the overlay correspond to the offloading spots and we characterize the logical links with attributes relevant to the allocation of the data transfers, including the capacity, travel time, and data leakage. The data leakage accounts for the proportion of vehicles that fail to deliver the data to the next offloading spot because of errors in the prediction of the direction of the vehicle or incidents. We showed the offloading overlay effectively decreases the complexity the main roads of France, as it provides a logical overlay with an exponential difference in the number of paths for given path travel times. 

% \paragraph{Feasibility assessment.}

% We evaluate the feasibility of data offloading on the road network by comparing it to data transfers using infrastructure-based networks such as the Internet. We design an allocation by formulating a linear programming model that maximizes the cost-benefit of offloading traffic on the road network compared to using infrastructure-based networks. We enable data replication at the offloading spots to guarantee reliable data transfers. We evaluate the allocation model on the main roads of France that feature actual road traffic counts with a conservative characterization of the capacity of the logical links of the offloading overlay. We show that the road network can accommodate concurrent data transfers with an aggregate capacity in the Petabyte range per week.

% \paragraph{Road resource allocation and management.}

% We use the centralized architecture and the offloading overlay to allocate offloading demands to the flows of vehicles represented by the logical links. We model the allocation as a max-min fairness model to guarantee fair allocation of competing demands and efficient utilization of the vehicular resources. We then translate the allocation output into forwarding rules at the offloading spots to select the data to load on stopping vehicles. We leverage the controller to implement mechanisms to ensure reliable data transfers. To this end, we use retransmission techniques at the offloading spots to send data again if it failed to be delivered to the expected offloading spot. Additionally, we consider redundancy techniques (\ie RAID~6) to decrease the amount of data to retransmit. We evaluate the throughput and fairness of data transfers on the roads of France with a more realistic characterization of the capacity of the logical links using elaborated traffic forecasting techniques. We show that the road network has the potential to offload several Petabyte of data per day, thus improving the results of our previous feasibility assessment.

% \paragraph{Offloading spot placement for vehicular cloud services.}

% We propose to extend the concept of offloading spots we present in the previous contributions, we propose to 
% Offloading spots storage and dematerialization 
% Placement of offloading spots that act as repositories that are part of a distributed cloud-like storage and sharing system
% Placement to jointly satisfy a maximum of user requests to store or retrieve files and distribute the files throughout the repository network using the movements of the mobile users between the repositories. Show the benefits of the placement to distribute files and satisfy user requests in a timely fashion using the bus movements of San Francisco.

% Placement of dematerialized offloading spots where vehicle often meet for durations long enough to accommodate transfers of large amounts of data


% PERSPECTIVES

The work presented in this thesis exploit the existing mobility of everyday entities, including private vehicles and bus transportation systems to transport data in replacement or in tandem with conventional data networks. 

Throughout the thesis, we followed two distinct but complementary research directions. The first direction consists in characterizing the vehicular mobility in terms of networking quantities. This allows the use of network management tools to handle and allocate the mobility as a network resource.
The offloading overlay relies on the movements of the vehicles between offloading spots to characterize its logical links in terms of capacity, delay, and loss. With this representation, we were able to implement the allocation of the data transfers, as well as mechanisms for reliable data transfers. To efficiently place the offloading spots in the context of the storage and sharing system, we leveraged a logical representation of the places the vehicles visit and characterized their movements between these places as networking quantities in terms of capacity and delay. Finally, in the case of the vehicular resource virtualization, we used a logical representation of the locations of the contacts between the vehicles and their durations to determine the optimal places to exchange large amounts of data. The tools we developed in this thesis are generic and broad, as they apply to any kind of entity as long as they are mobile.

The second direction we followed consists in controlling the vehicular resources, although they are random. At first, we used a dedicated infrastructure strategically placed or chosen to combine the movements of the vehicles together. This infrastructure mitigates the scale of the vehicular network and enhances its capacity by increasing the number of possible combinations of vehicle trajectories. This increases the number of logical paths the data can follow to reach its destination. This was the role of the offloading spots, that act as relays by passing data asynchronously between vehicles stopping at the offloading spots. The use of the offloading spots to control the mobility also enabled the implementation of mechanisms for reliable data transfers, which can be seen as logical paths followed by the data transported in turns by vehicles when traversing the successive offloading spots. In the case of the storage and sharing system, we controlled the vehicular resources by using the offloading spots as repositories to combine the movements of the users and distribute data among the repositories. Finally, in the case of the vehicular resource virtualization, we began to dematerialize the offloading spots into geographic areas where it is possible to combine the trajectories of vehicles. 

% CONCLUSION

The characterization of the mobility as network resources together with the offloading spot dematerialization to combine the trajectories of mobile entity have the potential of creating original and innovative value-added services for urban environments.
