\chapter*{Thesis abstract}
\vspace{-10pt}
In this thesis, we tackle mobility from a new perspective. We exploit the daily mobility of vehicles to create an alternative transmission medium. Our objective is to draw on the many vehicular trips taken by cars or public transports to overcome the limitations of conventional data networks such as the Internet. In the first part, we take advantage of the bandwidth resulting from the mobility of vehicles equipped with storage capabilities to offload large amounts of delay-tolerant traffic from the Internet. Data is transloaded to data storage devices we refer to as \textit{offloading spots}, located where vehicles stop often and long enough to transfer large amounts of data. Those devices act as data relays, \ie they store data it is until loaded on and carried by a vehicle to the next offloading spot where it can be dropped off for later pick-up and delivery by another vehicle. The vehicles allocated to the transport of data are selected according to their direction so as to meet the performance requirements of data transfers. We formulate the data transfer allocation problem as a linear programming model, which maximizes the cumulative throughput achieved by the transfers of offloaded data in a fair manner. To solve this model, we characterize the flows of vehicles traveling the road segments connecting adjacent offloading spots into network quantities such as capacity, delay, and loss. The resulting overlay network reduces the complexity of the road network topology and makes the data transfer allocation tractable. With actual traffic counts of the French roads, we show that the road network has the potential to offload several Petabytes of data per day. We further extend the concept of offloading spots according to two directions in the context of vehicular cloud services. In the first extension, we exploit the storage capabilities of the offloading spots to design a cloud-like storage and sharing system for vehicle passengers. The offloading spots act as repositories where users first upload their files, which are then replicated among the other repositories using the movements of the vehicles traveling between the repositories. To process the requests to store or retrieve a file in a timely fashion, we design a placement algorithm that determines the locations where the repositories can capture a maximum number of user requests before they expire. In the second extension, we dematerialize the offloading spots into pre-defined areas that feature high densities of vehicles and where vehicle-to-vehicle communications last long enough to transfer large amounts of data. We exploit these areas to migrate virtual machines between vehicles without involving intermediate offloading spots. The performance evaluation of the various works conducted in this thesis shows that everyday mobility of entities surrounding us enables innovative services with limited reliance on conventional data networks.        


\section*{Keywords}

Offloading, vehicular networks, resource management, virtualization, software-defined networking, delay-tolerant networking.
