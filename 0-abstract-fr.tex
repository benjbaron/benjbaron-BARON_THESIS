
\chapter*{Résumé de la thèse}

Dans cette thèse, nous appréhendons la mobilité sous un nouvel angle. Nous proposons d’exploiter la mobilité quotidienne articulée autour de véhicules pour créer un médium de communication ad hoc. Notre objectif est de tirer partie des trajets quotidiens effectués en voiture ou en transport en commun pour surmonter les limitations des réseaux de données tels que l’Internet. Dans une première partie, nous tirons partie de la bande passante que génèrent les déplacements de véhicules équipés de capacités de stockage afin de délester (\textit{offload}) en masse l’Internet d’une partie de son trafic. Les données sont détournées vers des équipements de stockage appelés points de délestage installés aux abords de zones où les véhicules s’arrêtent habituellement. Ces équipements agissent comme des points relais où les données sont entreposées avant d’être chargées et transportées sur un véhicule, et ce jusqu’au prochain point de délestage où elles pourront éventuellement être déchargées. La sélection des véhicules alloués au transport des données dépend de leur itinéraire mais aussi des contraintes qui caractérisent les données délestées. Nous formulons le problème de l’allocation des véhicules sous forme d’un modèle de programmation linéaire qui maximise le débit total résultant du transport des données sur le réseau routier tout en garantissant leur équité. Pour résoudre ce modèle, nous virtualisons les flots de véhicules circulant sur les segments de route reliant les points de délestage. Le réseau virtuel résultant permet de caractériser la mobilité en termes de bande passante, de délai et de taux de perte tout en réduisant la complexité de l’infrastructure de délestage. Nous proposons ensuite d’étendre le concept de point de délestage selon deux directions dans le contexte de services reposant toujours la mobilité des véhicules. Dans la première extension, nous proposons d’exploiter les capacités de stockage des points de délestage afin de concevoir un service de stockage et partage de fichiers offert aux passagers de véhicules. Les points de délestage se comportent comme des points de stockage où les passagers déposent leurs fichiers, qui sont ensuite répliqués au niveau des autres points de stockage en utilisant le mouvement des véhicules. Afin d’exécuter les requêtes des passagers pour déposer ou récupérer un fichier dans des délais contraints, nous proposons un algorithme qui détermine les emplacements des points de stockage de telle façon que chacun capture un nombre maximum de requêtes de passagers avant qu’elles n’expirent. Dans la seconde extension, nous dématérialisons les points de délestage en zones géographiques pré-définies. Ces zones se distinguent par leur concentration d’un grand nombre de véhicules où les communications entre véhicules durent suffisamment longtemps pour transférer de grandes quantités de données. L’évaluation des performances des différents travaux menés au cours de cette thèse montrent que la mobilité inhérente aux entités du quotidien permet la fourniture de services innovants avec une dépendance limitée vis-à-vis des réseaux de données traditionnels.

\section*{Mots-clefs}

Délestage de données, réseaux véhiculaires, gestion de ressources, virtualisation, \textit{software-defined networking}, \textit{delay-tolerant networking}.
