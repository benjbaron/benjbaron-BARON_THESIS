\chapter{Conclusion and perspectives}
\label{cha:conclusion}

\section{Summary of contributions and takeaways}

The work presented in this thesis exploits the existing mobility of everyday entities, including private vehicles and bus transportation systems to overcome the limitations of conventional data networks. We used the mobility in the context of various applications, including traffic offloading and cloud-based services. 

\paragraph{Massive data offloading.} 
We equipped vehicles with storage capacities connecting specific locations we referred to as offloading spots. The offloading spots avoid relying on the vehicles that make a direct trip from the source to the destination of the data transfer. The offloading spots help maximize the utilization of the combined storage of vehicles traveling the road segments connecting the offloading spots. To this end, they act as data relays where the trajectories of the vehicles making stops are composed into a single path followed by the data they carry.  
% The data then follows the path consisting of a sequence of offloading spots where the trajectories of independent vehicles are composed. 
In the following, we detail how we used the offloading spots to control and manage the composition of the movements of the vehicles to match the requirements of the data transfers.

\paragraph{Control and management of the offloading infrastructure.}
Offloading spots are also equipped with storage where data is buffered and asynchronously passed from one vehicle to another. The offloading spots are located where vehicles stop long enough to transfer large amounts of data. Examples of such locations include parking lots, shopping centers, and gas stations. We proposed an \acrshort{sdn} architecture with a centralized controller that has a holistic representation of the offloading infrastructure. The controller manages the resources of the offloading system using a logical representation of the dynamics of the road network. This representation is called the offloading overlay and results from a mapping algorithm that translates the movements of vehicles in terms of network quantities. The offloading overlay consists of logical links, each characterized by its capacity, delay, and loss rate. The controller leverages the offloading overlay to solve the vehicle flow allocation problem, which consists in selecting and allocating data to the vehicle flows connecting the offloading spots. The resolution of the allocation problem maximizes the utilization of the road resources. The output of the allocation is translated into a set of rules the controller uses to configure the offloading spots. These rules indicate which vehicles an offloading spot should select and which data to load onto the vehicles. Our results on the main roads of France with real traffic counts showed that the centrally controlled architecture can achieve an efficient and fair allocation of concurrent data transfers between major cities in France.

\paragraph{Vehicular storage and sharing system.}
We exploited the storage of the offloading spots to turn them into repositories where mobile users can store and retrieve files. The system relies on a collection of repositories and the movements of the vehicles between them to distribute the files among the repositories. We proposed a placement algorithm to determine the optimal locations of the repositories so each repository can satisfy the user requests to store or retrieve a file. The placement of the repositories also considers the movements of the vehicles connecting each pair of repositories so the files uploaded to the system can be synchronized without relying on the existence of a data network. We showed that the distribution of files using the existing mobility of the vehicles helps increase the success rate of user requests.

\paragraph{Virtual vehicular network.}
We dematerialized the offloading spots in the context of a vehicular network where the vehicle resources, including storage, but also processing or sensing are virtualized with virtual machines. We studied the virtual machine migration through vehicle-to-vehicle communications. In this context, the offloading spots are areas where vehicles come in contact often and for long periods compared to other areas. We showed that restricting the virtual machine migrations to these areas improves the success rate of the migrations.

\paragraph{Main challenges.}
During this thesis, I had to find data sources through people I contacted in different institutions and companies. While most of these attempts were unsuccessful, I managed to find relevant local data following recent open-data initiatives, as an increasing number of institutions provide publicly available data. Once I obtained the data, I had to find ways to analyze it in order to characterize it into network quantities. Unfortunately, most of the open-source traffic simulators (SUMO~\cite{behrisch2011sumo} or MatSim\footnote{\url{http://www.matsim.org/}}) were unsuited for the analysis of the road networks. I had to develop my own set of tools to perform the analysis and characterization of the networks necessary for the evaluations. Finally, I had to study and understand the relevant body of work borrowed from transportation research to infer flows of vehicles from traffic counts and characterize them into network quantities.

To summarize, the key takeaways from this thesis are:
\begin{itemize}

    \item The vehicles in circulation are an untapped potential for data transportation. We illustrated this assertion with simulations of vehicles transporting data on French roads and showed that the road network has the potential to daily offload several Petabytes of data.
    
    \item A dedicated infrastructure increases the overall capacity of the system by composing the trajectories of the vehicles and configuring the path followed by the data.

    \item A centralized control of the infrastructure with a holistic representation of the vehicle flows enables efficient management and allocation of the road resources to data transfers.
    
    \item A logical representation of the vehicular resources into network quantities mitigates the scale of the road network and makes the data transfer allocation tractable. 
    
    \item The mobility of everyday entities gives the potential to new actors operating mobile entities (\eg vehicle owners or bus transit agencies) to provide value-added services with limited reliance on conventional data networks. We illustrated this assertion with our offloading system and the extensions that create vehicular cloud services.
    
\end{itemize}

\section{Perspectives}

\subsection{Short-term perspectives}

\paragraph{Offloading spot placement.}
In our data offloading, we created a realistic deployment of charging stations. We emulated charging demands issued within cities we weight according to their sizes. The charging stations are placed so as
to satisfy these demands. As a result, each charging station is located within the range of the surrounding electric vehicles. One could consider the charging demands made by vehicles while on the move. Some work considered charging demands issues by the traffic flows themselves~\cite{hodgson1990flow}. These deployment strategies make sense in the context of charging station networks such as the one consisting of the superchargers operated by Tesla. 
Instead of considering the charging demands, one could propose an alternative deployment strategy derived from the requirements of the data offloading. Similar to the placement algorithms we proposed in the two vehicular cloud services, one could optimize the placement of the offloading spots to maximize the capacity resulting from the combined vehicle storage. One could adapt the algorithms proposed in the context of the throwboxes work~\cite{zhao2006capacity} to determine the optimal number of offloading spots and their locations to capture the maximum flows of vehicles. We would need to estimate the flows of vehicles between pairs of origin and destination on the road network. A possible way of estimating this origin-destination matrix consists in discretizing the road network space in a collection of areas and using real traffic counts to derive the volume of vehicles traveling between these areas.

\paragraph{Dynamic offloading system.}
The objective of our work was to assess the feasibility of offloading data over the road network with regard to its capacity and realization. We proposed models and evaluations based on datasets giving for each road segment, real traffic counts averaged over a year. As we discussed in Section~\ref{sec:choice-traffic-counts}, using averaged traffic counts limits the effects of errors due to equipment failures or missing data. As a result, we assumed that vehicles visit the offloading spots following a Poisson distribution, according to the rate we derived from the origin-destination matrix calculation. 

In reality, the vehicular traffic is erratic and varies according to seasons, month or time of day. This is expected to affect our offloading system as data has more chances to pile up at offloading spots.To complete our work, one could study the impact of the traffic dynamics with regards to the dimension of the offloading spots and evaluate the resulting capacity. One way to do this is to use the recent work~\cite{fleischer2007quickest} that built on Ford and Fulkerson~\cite{ford2015flows} to study the allocation of flows over time using time-expanded graphs. 

In the case of data offloading, multiple edges connecting the offloading spots would take into account the discretized variations of the road traffic between the offloading spots (\eg every 15 minutes). We would then need to estimate the dynamic road traffic between pairs of offloading spots, using work from transportation research~\cite{cascetta1993dynamic,bierlaire2004efficient}. The dynamic allocation of the data transfers would give a more realistic realization of the offloading infrastructure and would allow us to dimension the system, in particular, the storage needed at the offloading spots to avoid data losses due to overflows.

\paragraph{Comparison with conventional data networks.}
In our evaluations of the data offloading process, we evaluated its performance in terms of achieved throughput. When evaluating and comparing the offloading process with traditional data networks such as the Internet, several performance metrics such as average throughput, transfer cost, and energy efficiency are relevant to assess the performance of transfers of massive amounts of data. However, the following limitations prevented us from comparing both systems. Firstly, we need to have equivalent settings to plan transfers of bulk data on both sides, including geographical coverage, the amount of data to transfer, the extent of the use of multi-path routing, and current utilization of the network. In particular, we need to be able to use the full capacity of a data network, which varies from one network to the other (\eg Renater uses mostly 10G links\footnote{\url{https://pasillo.renater.fr/weathermap/weathermap_metropole.html}} and ESnet uses mostly 100G links\footnote{\url{https://www.es.net/}}), and from one destination to the other (\eg Paris is better connected than Brest in the Renater network). Secondly, our current implementation of the offloading process does not take into account dynamic road traffic. As a result, the storage needed at the offloading spots cannot be properly provisioned, which hinders the estimate of the cost and energy needed for transferring data. Finally, while some works give estimates of the cost of the bandwidth~\cite{laoutaris2013delay,jin2016optimizing}, the cost of transferring bulk data through an \acrfull{isp} is not properly defined and varies a lot depending on the \acrshort{isp} and the amount of data to transfer.

\subsection{Medium- and long-term perspectives}

\paragraph{Drayage system.}\index{drayage}
In this thesis, we mainly focused on the feasibility of the offloading system between the edge offloading spots of the data transfers. In particular, we assumed that the transloading of data between the edge offloading spots and endpoints of the data transfers is operated by a drayage system that does not impact the performance of the system.

In Section~\ref{sec:discuss}, we already discussed ways to realize the drayage system using dedicated lines or dedicated vehicles to transfer the data between the endpoints and the edge offloading spots. Since these solutions are costly and require a dedicated infrastructure, they would not fit with the motivations for this thesis to leverage existing mobility. Instead, we could exploit the work we presented in the extensions to build a drayage system by combining the movements of vehicles that make shorter journeys (\eg daily commute from or to the workplace). 

Combining the mobility of vehicles of different scales would create a multi-tiered architecture involving intermodal end-to-end data transfers, ranging from large-scale (\eg country-wide) to local (\eg within a city) data transfers operating seamlessly by the already-existing mobility of the vehicles. Following our extensions, the local data transfers would be done by combining the movements of vehicles carrying the data. The combination could also rely on dedicated facilities (\eg bus stops) equipped with storage acting as offloading spots to pass the data from one vehicle to another or on specific geographical areas where vehicles come in contact for long durations to transfer large amounts of data.

\clearpage
\paragraph{Offloading spot dematerialization.}
We dematerialized the offloading spots in the extension of the thesis, such that they represent areas where vehicles can exchange large amounts of data synchronously while in contact. One could implement a similar approach in the context of the vehicular data offloading. Offloading spots could be made floating and the movements of vehicles combined as long as in contact. The data could be then passed directly between vehicles, \ie without being buffered at stationary intermediate nodes. Note that this approach is similar to the strategies we reviewed in Section~\ref{sec:indirect-sync-anchored} that synchronously compose the trajectories of entities at pre-defined locations. The main difference with our approach is that we exploit the capacity available through contact between entities to transfer large amounts of data. To this end, one could leverage the logical representation we used to abstract the mobility of the vehicles (\eg the offloading overlay). The nodes in the logical representation would refer to the locations where vehicles meet. The logical links would result from the aggregation of the movements of the vehicles and their characterization with networking quantities such as capacity and delay.

The main difference with our approach and those proposed in the state-of-the-art is that a central controller could leverage this logical representation to allocate reliable data transfer. The data would follow a logical path that consists of the vehicles that successively transport the data by exchanging data at the nodes defined in the logical representation. To realize the logical data path, the controller could install states embedded in the vehicles that trigger specific once in contact with other vehicles. These actions are similar to forwarding decisions that can depend  on the direction of the encountered vehicles for instance. Finally, the controller could take into account the current and future mobility patterns of the vehicles to dynamically update the logical representation of the system and the forwarding states at the vehicles resulting from the re-allocations of the data transfers.

\paragraph{Vehicle resource virtualization.} 
In the vehicular resource virtualization extension, we virtualized the resources of the vehicles to allow multiple service providers to deploy geo-distributed services that leverage the movements of vehicles. One can combine this work to the logical characterization of the vehicle movements we proposed to manage and allocate by a centralized architecture. We believe that together, these work have the potential to bring original and innovative value-added services for urban environments, especially in the context of the Fog Computing paradigm~\cite{bonomi2012fog}. 

One could involve the use of vehicles equipped with storage, sense, and compute capabilities to provide a densely distributed substrate to help collect and transport sensed data, but also to perform real-time local analysis and decision-making~\cite{bonomi2012fog,aubry2014crowdout}. To make vehicles operational from the point of view of the Fog Computing paradigm, the mobility of the vehicles has a strong impact on the type of data they generate, aggregate, or collect. One could use our proposal of a controller to allocate and control the resources offered by the vehicles. The controller can help leverage the combined compute and storage capabilities of the vehicles by avoiding the need of collecting data through a control channel (\eg cellular connection). Instead, the vehicles could aggregate the data they generate or collect from other vehicles they meet with at dematerialized offloading spots. The vehicles could then process on-the-go the data they carry, while they are moving between the offloading spots. 
%The offloading spots allow the exchange of data and the composition of processing between vehicles. 
The data, once aggregated and processed by the vehicles, can then be transported to central backend systems for further processing and analysis.


